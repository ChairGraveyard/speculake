

\section{Triggers}
%Easy answer: custom program that mistrains branch predictor

So far, we have described using a custom program as a trigger, which performs a
pattern of indirect jumps to mistrain the indirect branch predictor, leading the payload program
to its speculative entry point. In this
section, we describe alternative triggers, including using benign programs
already on the system, and recent Spectre variants.

\subsection{Benign Program Triggers}
\label{subsec:openssl}

Custom trigger programs that are installed with malicious payload programs may
be easy for an analyst to pair up and analyze. As an alternative to trying to
hide the trigger program from the analyst, ExSpectre can use \emph{benign}
programs already installed on the system as a trigger.

For example, if a benign application makes a series of indirect jumps---thus training
the indirect branch predictor---an ExSpectre payload can make similar indirect
jumps leading up to its speculative entry point. The payload's speculative entry
point will be determined by the benign application, but may be even more
difficult for an analyst to discover, as now the ExSpectre trigger could be
\emph{any} application running concurrently on the system.

\paragraph{OpenSSL}
We experimented using the OpenSSL library as a potential benign trigger
application, as its source code has a gratuitous use of function pointers which
compile to indirect jumps. In addition, it has many complicated code paths that
can be easily selected by remote clients through their choice of cipher suite.
This allows a remote attacker to trigger ExSpectre malware on a server running a
(benign) TLS stack supported by OpenSSL, simply by making a large number of TLS
connections with a specific cipher suite. We describe our implementation using
OpenSSL as a trigger in Section~\ref{subsec:openssl-impl}.

\medskip

In addition, an adversary could use a benign application (like OpenSSL) to
\emph{communicate} information covertly to the malicious payload program. For
example, with OpenSSL, the attacker could have a pair of cipher uncommon cipher
suites, and using one results in communicating a 1-bit, while the other
communicates a 0-bit to the payload. To receive this, the payload would have to
do indirect jump patterns corresponding to both cipher suites, with the
corresponding speculative entry points shifting in the appropriate bit. Thus, an
adversary can communicate remotely (over a network) to the payload program
indirectly via a benign application.

We observe that communication could also go in the other direction: from
the payload program back to the remote adversary, also via a benign application
intermediary. The payload program could influence the performance of the benign
application, and the adversary could time responses from the benign application
to receive covert information from the malicious payload.

%Could also be any program on the system, and potentially remote!




\subsection{Speculative Buffer Overflow}

\FigSpectreOne

In addition to using separate trigger programs, ExSpectre can also use
\emph{trigger inputs} to a payload program to initiate its malicious behavior.
While existing fuzzers and symbolic execution tools can discover traditional
input triggers, we can leverage other Spectre variants to obfuscate speculative
gadget access. 

We make use of Specualtive Buffer Overflows (SBO) to redirect control flow
to a speculative entry point specified by user input. Using Spectre variant 1.1 
we can design a program that takes seemingly arbitrary user input and performs complete 
bounds checks such that evaluated code is safe, however a direct branch can
be bypassed speculatively given significant lead up with consistent input 
to train the branch predictor (see Figure~\ref{fig:spectre-one}). This allows the input
to become the trigger. 

A direct branch followed by a store operation with user controlled index and
value can be coerced into overflowing a local array. 
As seen in code snippet from Kiriansky and Waldspurger~\cite{kiriansky2018speculative}:
\begin{lstlisting}
    if (y < lenc)
        c[y] = z;
\end{lstlisting}
with user controlled 
\texttt{y} and \texttt{z} and sporadically uncached \texttt{lenc}, an attacker 
can \textit{speculatively} overflow array \texttt{c}
to overwrite a return address, function pointer, etc. and redirect control flow.

 

We make use of this in a willing payload such that user input can intentionally 
mistrain the branch predictor. The speculative entry point is also chosen by the 
trigger in this scenario as \texttt{z} contains the address of the 
speculative entry point, allowing the attacker to create a ROP style speculative 
execution path through the payload.


\subsection{Speculative Store Bypass}

Speculative Store Bypass (SSB) can similarly be used to construct an internal trigger
using race conditions on a store and a subsequent load that the cpu believes 
will not alias but in reality do~\cite{spec-store-bypass}.

To redirect control flow a store to a function pointer or indirect branch 
target pointer could be speculated to a stale value. While the program never 
executes at this stale address in reality, the speculative gadgets have a 
real effect on shared resources. This attack again allows ROP style 
chains to execute speculative gadgets.

The speculative primitive is shared by all versions of Spectre as an upper
bound on what can be done speculatively. What is required to adapt a spectre 
vulnerability into the \speculake technique is an attacker controlled method of 
triggering control flow redirection in a willing payload to a speculative entry point. 

%\subsection{Complex Triggers}
%When considering packers and crypters used in modern malware it is not uncommon to see 
%a malicious sample packed using multiple stages with different unpacking conditions. 
%Whether that be red pills, or environent checks, or network triggers, the reliance is on
%complexity to fool any reverse engineer out of finding the correct conditions to 
%release the full payload.
%
%Similarly \speculake can be instrumented to use more than one trigger. Multiple stages 
%can be designed used any combination of crafted or benign trigger programs. Only when all
%stages have been run to completion in order will the malicious payload be revealed.  
%In conjunction with this each stage cna use a randomized ISA to a common emulator, or 
%they can use differnt emulators all together. 
%
%Alternatively each stage could be used to decrypt the next AES key-schedule to decrypt 
%a subsequent layer. Using this model we can instrument \speculake malware as a crypter 
%with the keys hidden in data or dead code which static and dynamic reverse engineering 
%methods will overlook. 
%

%%%%%%%%%%%%%%
