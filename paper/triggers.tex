

\section{Triggers}
%Easy answer: custom program that mistrains branch predictor

So far, we have described using a custom program as a trigger, which performs a
pattern of indirect jumps to mistrain the indirect branch predictor, leading the
payload program to its speculative entry point. In this section, we describe
alternative triggers, including using benign programs already on the system, and
recent Spectre variants.

\subsection{Benign Program Triggers}
\label{subsec:openssl}

Custom trigger programs that are installed with malicious payload programs may
be easy for an analyst to pair up and analyze. As an alternative to trying to
hide the trigger program from the analyst, ExSpectre can use \emph{benign}
programs already installed on the system as a trigger.

For example, if a benign application makes a series of indirect jumps---thus
training the indirect branch predictor---an ExSpectre payload can make similar
indirect jumps leading up to its speculative entry point. The payload's
speculative entry point will be determined by the benign application, but may be
even more difficult for an analyst to discover, as now the ExSpectre trigger
could be \emph{any} application running concurrently on the system.

\paragraph{OpenSSL}
We experimented using the OpenSSL library as a potential benign trigger
application, as its source code has a gratuitous use of function pointers which
compile to indirect jumps. In addition, it has many complicated code paths that
can be easily selected by remote clients through their choice of cipher suite.
This allows a remote attacker to trigger ExSpectre malware on a server running a
(benign) TLS stack supported by OpenSSL, simply by making a large number of TLS
connections with a specific cipher suite. We describe our implementation using
OpenSSL as a trigger in Section~\ref{subsec:openssl-impl}.

\medskip

In addition, an adversary could use a benign application (like OpenSSL) to
\emph{communicate} information covertly to the malicious payload program. For
example, with OpenSSL, the attacker could have a pair of uncommon cipher suites,
where using one results in communicating a 1-bit, while use of the other
communicates a 0-bit to the payload. To receive data, the payload would have to
do indirect jump patterns corresponding to OpenSSL code for processing both
cipher suites, with the corresponding speculative entry points shifting in the
appropriate bit. Thus, an adversary can communicate remotely (over a network) to
the payload program indirectly via a benign application.

We observe that communication could also go in the other direction: from the
payload program back to the remote adversary, also via the benign application
intermediary. The payload program could influence the performance of the benign
application, and the adversary could time responses from the benign application
to receive covert information from the malicious
payload~\cite{schwarz2018netspectre}. This would allow the malicious payload to
operate \emph{entirely speculatively}, without assistance from the real world
for keeping state.

%Could also be any program on the system, and potentially remote!




\subsection{Speculative Buffer Overflow}
\label{subsec:spec-buffer-overflow}

In addition to using separate trigger programs, ExSpectre can also use
\emph{trigger inputs} to a payload program to initiate its malicious behavior.
While existing fuzzers and symbolic execution tools can discover traditional
input triggers, we can leverage other Spectre variants to obfuscate our
triggers.

To do this, we use Specualtive Buffer Overflows
(SBO)~\cite{kiriansky2018speculative} to redirect control flow to a speculative
entry point specified by user input. We design a payload program that takes
arbitrary user input and performs appropriate bounds checks to ensure no
traditional control flow violations could be exploited. However, using the
Spectre~1.1 variant, control flow can still be violated speculatively, allowing
the adversary to force a speculative entry point based entirely off the input
provided. This allows the trigger to be an input, potentially even provided over
a network if the payload program accepts network data. Traditional symbolic
execution and code coverage fuzzers will be unable to discover this trigger
input, as they do not model the speculative state of the CPU.

%such that evaluated code is safe, however a direct branch can
%be bypassed speculatively given significant lead up with consistent input 
%to train the branch predictor (see Figure~\ref{fig:spectre-one}). This allows the input
%to become the trigger. 

To create an SBO-triggered payload program, we make the following code pattern,
as seen from Kiriansky and Waldspurger's Spectre~1.1
description~\cite{kiransky2018speculative}:
\begin{lstlisting}
    if (y < lenc)
        c[y] = z;
\end{lstlisting}
With user controlled 
\texttt{y} and \texttt{z} and sporadically uncached \texttt{lenc}, an attacker 
can \textit{speculatively} overflow array \texttt{c}
to overwrite a return address (or function pointer, etc) and redirect control
flow. Note that the bounds check on \texttt{y} will ensure that this program
will not actually allow a buffer overflow to occur, but the attacker can
nonetheless use this to influence a speculative entry point based on their
choice of \texttt{y} and \texttt{z}.


We make use of this pattern in a willing payload such that user input can
intentionally mistrain the branch predictor by repeatedly sending valid
(in-bounds) values of \texttt{y} before sending a value that would overflow the
bounds of \texttt{c}. The speculative entry point is also chosen by the trigger
in this scenario as \texttt{z} contains the address of the speculative entry
point, allowing the attacker to create a ROP-style speculative execution path
through the payload. We describe our implementation of this in Section~\ref{tk}.


\subsection{Speculative Store Bypass}

Speculative Store Bypass (SSB) (Spectre variant 4) can similarly be used to
construct an internal (input-based) trigger using the CPU's speculative
load-store forwarding~\cite{spec-store-bypass}. In a speculative store bypass,
the CPU incorrectly speculates that a store will not alias with a future load,
and uses a stale (wrong) value for the result of the speculative load.


To redirect control flow, a payload program could use a function pointer or
indirect branch target register as the destination for a speculative store
bypass, causing the CPU to use a stale value to speculatively determine where it
would go. The stale value could be controlled by a previous unrelated input,
allowing an adversary to specify the speculative entry point in a carefully
crafted data input. While the program never
executes at this stale address in reality, the CPU will briefly speculatively
execute there, enabling \speculake payloads. Like the speculative buffer
overflow, this trigger also allows ROP style chains to execute a series of
speculative gadgets.


%The speculative primitive is shared by all versions of Spectre as an upper
%bound on what can be done speculatively. What is required to adapt a spectre 
%vulnerability into the \speculake technique is an attacker controlled method of 
%triggering control flow redirection in a willing payload to a speculative entry point. 

%\subsection{Complex Triggers}
%When considering packers and crypters used in modern malware it is not uncommon to see 
%a malicious sample packed using multiple stages with different unpacking conditions. 
%Whether that be red pills, or environent checks, or network triggers, the reliance is on
%complexity to fool any reverse engineer out of finding the correct conditions to 
%release the full payload.
%
%Similarly \speculake can be instrumented to use more than one trigger. Multiple stages 
%can be designed used any combination of crafted or benign trigger programs. Only when all
%stages have been run to completion in order will the malicious payload be revealed.  
%In conjunction with this each stage cna use a randomized ISA to a common emulator, or 
%they can use differnt emulators all together. 
%
%Alternatively each stage could be used to decrypt the next AES key-schedule to decrypt 
%a subsequent layer. Using this model we can instrument \speculake malware as a crypter 
%with the keys hidden in data or dead code which static and dynamic reverse engineering 
%methods will overlook. 
%

%%%%%%%%%%%%%%
