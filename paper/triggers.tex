

\section{Trigger Programs}
Easy answer: custom program that mistrains branch predictor

\subsection{Benign Triggers}
Could also be any program on the system, and potentially remote!


\subsection{Complex Triggers}
When considering packers and crypters used in modern malware it is not uncommon to see 
a malicious sample packed using multiple stages with different unpacking conditions. 
Whether that be red pills, or environent checks, or network triggers, the reliance is on
complexity to fool any reverse engineer out of finding the correct conditions to 
release the full payload.

Similarly \speculake can be instrumented to use more than one trigger. Multiple stages 
can be designed used any combination of crafted or benign trigger programs. Only when all
stages have been run to completion in order will the malicious payload be revealed.  
In conjunction with this each stage cna use a randomized ISA to a common emulator, or 
they can use differnt emulators all together. 

Alternatively each stage could be used to decrypt the next AES key-schedule to decrypt 
a subsequent layer. Using this model we can instrument \speculake malware as a crypter 
with the keys hidden in data or dead code which static and dynamic reverse engineering 
methods will overlook. 


%%%%%%%%%%%%%%
