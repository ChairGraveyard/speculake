
\section{Implementation and Evaluation}

\subsection{Turing Machine}

% Target_fn returning next step given access to current state in global vars
%   - busy beaver
%   - cache page speculation on repetetive cycles. 
%       - mis-predict w/in speculative world 

% speed (complete smaller busy beaver?)
% against reference python implementation?

\subsection{AES Decryption}
The speculative world is able to take advantage of the AES-NI instructions to
decrypt messages. In order to avoid detection (as well as reduce the computation
that needs to be done in the speculative world) the key expansion can be done
ahead of time. However, the key schedule does have a structure that could be
detected by a reverse engineer. Thus 11 random round keys are selected. This
obfuscated key schedule makes it more difficult for a reverse engineer to
determine that there is encrypted content. Additionally since the content is
only accessed when the correct trigger program is running the challenges facing
a reverse engineer are great. We discuss further methods for obfuscating specualtive
decryption in section~\ref{subsec:future-work}. 


We have implemented the AES decryption as a fundamental piece of the \speculake model,
and we demonstrate that the speed at which information can be decrypted using the speculative 
primitive is dependent on the configuration of the \speculake model. Specifical,
the channel width and the required minimum signal strength determine the maximum
throughput that can be acheived. As demonstrated in Figure~\ref{fig:spec_bangwidth} 
between 6 and 8 bits channel width provides an idea of the maximum speed at which the 
speculative primitive can be used with AES-NI instructions in the payload. 

While the \speculake model acheives only moderate decryption speed, the 
critical computation of the decryption is done in the speculative world. 

\FigSpecBandwidth
% implementation 
%   Key Schedule expansion / obfuscation
%
% you can speculatively execute AES instructions 

\subsection{Virtual Machine}


\subsubsection{SPASM}
\label{subsubsec:spasm}
Constructing an emulator making use of the speculative primitive requires  
a trade off in expresive capability versus speed. 
We have implemented SPASM as a model using two pseudo-registers, and six 
bit instruction length which allows for a relatively direct programming model 
in which structured values can be entered into memory locations before making 
a systemcall.

To acheive a balance with speed the number of bits in each instruction is both 
fixed and minimized.  A variable length instruction would require that the prime
and probe stage search the maximum number of bits on each round, and each 
aditional bit doubles the search space that the prime and probe stage must 
traverse.  So every bit shorted effectively doubles the throughput of the 
emulator, and there is effectively no advantage to allowing variable length 
instructions. 

Using this model we have implemented multiple example programs that can be run
as encrypted payloads in an \speculake payload. A \textit{HelloWorld} program 
that prints to stdout. A \textit{FizzBuzz} program that demonstrates control 
flow and arithmetic operations while printing to stdout. And finally,
a \textit{ReverseShell} program that opens and connects to a socket before  
duplicating I/O and executing a local shell. 

Expected performance for a given SPASM binary will vary given multiple factors, 
the most important of which is the prime and probe redundancy. This can be tuned
based on the specific trigger, as it is used to establish confidence that a signal 
has been identified in the indicess returned from the "speculative world". 
Thus accomplishing a task using SPASM generally equates to:

\begin{lstlisting}
    Redundancy * Probe_Space * Num_Instr
\end{lstlisting}

For example, the \textit{ReverseShell} program makes six system calls to open a socket,
connect to it, duplicate I/O, and execve a specified program. This requires 355
SPASM instructions to complete. The probe space for the 6-bit SPASM ISA requires 
a traversal of an array od size 64. When using a redundancy of 256 to establish
confidence in the signal the entire \textit{ReverseShell} program takes under 7 
seconds to run once triggered. 

\FigSpasmModel

\subsubsection{ARM}
\label{subsubsec:arm}
While the custom emulator that we developed gives a much higher level abstraction to 
a developer, it still requires programs to be written in custom assembly 
language. To demonstrate that the extensibility of this model we also implemeted a wrapper
around an ARM emulator, such that programs can be written in a high level languages 
such as C or C++, statically compiled to binaries for the ARM emulator, and deployed 
as an encrypted data sections to be decrypted at run-time.  


% ARM Wrapper
%   - memory
%   - single instruction vs block

We chose to implement a wrapper around ARM as it has a 32-bit fixed instruction
length. However this causes a significant hit to the speed at which operations 
can be performed as the probe space is increased by a factor of four in 
comparison with SPASM,  and we must complete 4 of these transactions to 
retreive a single ARM opcode. Trading more expressive operations for speed
results in a 16x slowdown when considering per-instruction throughput. 

\begin{lstlisting}
    4 * Redundancy * Probe_Space * Num_Instr
\end{lstlisting}

\subsection{OpenSSL Trigger}

% extract jump patterns
% sufficiency checks 
%
% trigger strength
%   - future work developing tools to determine sufficiency and strength
%
% OPENSSL TRIGGERING REVERSE SHELL SPASM PAYLOAD

%%%%%%%%%%%%%%
