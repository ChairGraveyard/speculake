
\section{Introduction}


Modern CPU designs use speculative execution to maintain high instruction
throughput, with the goal of improving performance. In speculative execution,
CPUs execute likely future instructions while they wait for other slower
instructions to complete. When the CPU's guess of future instructions is correct, the
benefit is faster execution performance. When its guess is wrong, the CPU
simply discards the speculated results and continues executing along the true path.


Previously, it was assumed that speculative execution results remain invisible
if discarded, as careful CPU design maintains strict separation between
speculative results and updates to architectural state. However, recent research
has revealed side channels that violate this separation, and researchers have
demonstrated ways to exfiltrate results from speculative computation. Most
notably, the Spectre vulnerability allows attackers to leak information from
purposefully mis-speculated branches in a victim process~\cite{spectre}. The
Meltdown vulnerability uses speculative results of an unauthorized memory read
to sidestep page faults and leak protected memory from the
kernel~\cite{meltdown}. Both of these vulnerabilities focus on extracting secret
data from a process or operating system. Recent follow-up work has revealed
other Spectre ``variants'',  including speculative buffer overflows, speculative
store bypass, and using alternative side channels besides the
cache~\cite{kiriansky2018speculative, spectre_ng}. In addition, several attacks
have leveraged Spectre to attack Intel's SGX~\cite{chen2018sgxpectre,
spectre_sgx, foreshadow}, and perform remote leakage
attacks~\cite{schwarz2018netspectre}.

\medskip

In this paper, we explore another attack enabled by speculative execution:
\speculake, which \emph{hides computation} within
the ``speculative world''. Taking advantage of the CPU's speculation to secretly
perform computation,
we can produce binaries that thwart existing reverse engineering
techniques. Because the speculative parts of a program never ``truly'' execute,
we can hide program functionality in the unreachable dead code in a program.
Even a full instruction trace, captured by a hardware debugger or software
emulator, will be unable to capture the logic performed speculatively.
This technique could lead to sophisticated malware that hides its behavior
from both static and dynamic analysis.


Existing malware use several techniques to evade detection and
make it difficult for analysts to determine payload behavior of reported malware. 
For example, binary \emph{packers} or \emph{crypters} encode an executable payload as
data that must be ``unpacked'' at runtime, making it difficult to tell
statically what a program will do~\cite{malware-packers}. Malware may also use
\emph{triggers} that only run the payload when certain conditions are present, preventing
it from executing when it is inside an analysis sandbox or 
debugger~\cite{balzarotti2010efficient,red-pill}.

However, with some effort, these existing malware techniques can be defeated.
Analysts can use
dynamic execution to unpack malware and reveal its
behavior~\cite{balzarotti2010efficient}, and can use symbolic execution or code
coverage fuzzers to determine the inputs or triggers that will reveal malicious
behavior~\cite{moser2007exploring,schwartz2010all,wang2017angr,egele2012survey}.


\speculake provides a new technique to malware authors, allowing them to hide
program functionality in code that appears to not execute at runtime by
leveraging Spectre as a feature~\cite{necessaryBlog}. This technique defeats
existing static and dynamic analysis, making it especially difficult for malware
analysts to determine what a binary will do.

\FigHighLevel

At a high level, \speculake consists of two parts: a payload program, and a
trigger. The trigger can take the form of a special input (as in typical
malware), or an unrelated program running on the same system. When run without
the trigger, the payload program executes a series of benign operations, and
measures a cache-based side channel\footnote{We note that other side channels
could be used in place of a cache}. Once the trigger activates---either
by the attacker providing specially crafted input, or the trigger program running---it causes
the CPU to briefly \emph{speculatively} execute from a new target location inside the
payload program.

This target location can be in a region that is neither read
nor executed normally by the payload program, making this logic effectively dead code to
any static or dynamic reachability analysis. After the CPU discovers the
mis-speculation at the target location, it will discard the results and continue
executing from the correct destination. However, this still gives the payload program a
limited speculative window where it can perform arbitrary computation, and can
communicate results back to the ``real world'' via a side
channel. Figure~\ref{fig:high-level} shows the variant of \speculake that uses a
trigger program to mis-train the CPU's indirect branch predictor, causing the
payload program to briefly execute a hidden target function speculatively.


%At a high level, \speculake consists of two parts: a payload program, and a
%trigger program. When run by itself, the payload program executes a pattern of
%indirect jumps and measures a cache-based side channel\footnote{We note that other
%side channels could be used in place of a cache}. While the trigger program is
%not running, the payload program is effectively inactive (potentially performing
%a decoy operation). When the trigger program runs, it executes a similar
%pattern of indirect jumps (with similar source and destination addresses as the
%payload program), effectively training the CPU's branch predictor to the jump
%pattern performed by the trigger program.
%
%Importantly, the trigger program and payload program's indirect jump patterns
%diverge on the destination of their final jumps. However, because the trigger
%program has trained the CPU's branch predictor, the CPU speculates that the
%payload program will continue following the pattern of the trigger program,
%causing it to execute a \emph{speculative gadget} in the payload
%program. This execution path is shown in Figure~\ref{fig:high-level}.
%
%We emphasize that this gadget is in a region that is neither
%read nor executed by the payload program, and after the CPU discovers the
%mis-speculation in the payload program, it will discard the results from the
%speculative gadget and continue executing from the correct destination. However,
%the payload program has a limited speculative window where it can perform
%computation, and can communicate results back to the ``real world'' payload 
%program via a side channel.

%Figure~\ref{fig:high-level}





%While it's possible for the trigger and payload programs to be bundled in a
%single program, an analyst aware of \speculake could easily find the speculative
%gadget in the payload program based on the jump pattern in the trigger program.
%However, if the trigger and payload are kept separate, the analyst has a much
%harder job and must identify both.


It is also possible for a trigger program to be a
\emph{benign} program already on the victim's computer. We
show this using the OpenSSL library as a benign trigger
program in Section~\ref{subsec:openssl}, activating a malicious payload program
when an adversary repeatedly connects to the infected OpenSSL server using a
TLS connection with a specific cipher suite.


%If the trigger program is another benign
%program on the system, the analyst has the difficult task of identifying which
%program, library, or even operating system component is responsible for training
%the CPU's branch predictor, and finding the specific set of jumps that occur at
%runtime. To make matters worse, the payload program can include dummy jump
%patterns that are effectively dead ends for the analyst, as they do not
%ultimately call the hidden speculative gadget.

We also show it is possible to obviate the trigger program entirely, and
instead use \emph{trigger inputs}, which are data inputs the attacker provides
directly to
the payload, causing the CPU to speculatively execute
at the attacker's chosen address. Unlike traditional malware input triggers, these inputs
cannot be inferred from the payload binary using static analysis or symbolic
execution, as the logic these triggers activate happen speculatively in the
CPU, which existing analysis tools do not model. We describe this technique in
more detail in Section~\ref{subsec:spec-buffer-overflow}.

Simulating or modelling the speculative execution path is a difficult task for a
program analyst hoping to reverse engineer an \speculake binary. First, the
analyst must reverse engineer and accurately model the closed-source proprietary
components of the target CPU, including the branch predictor, cache hierarchy,
out-of-order execution, and hyperthreading, as well as taking into account the
operating system's process scheduling algorithm. In contrast, the \speculake
author only has to use a partial model of these components and produce binaries
that take advantage of them, while the analyst's model must be complete to
capture all potential \speculake variants. Second, the analyst must run all
potential trigger programs or inputs through the simulator, including benign
programs with real world inputs. Both of these contribute to a time-consuming and expensive
endeavor for would-be analysts, giving the attacker a significant advantage.

% Results summary
In order to study the potential of \speculake, we implement several example
payload programs and triggers variants, and evaluate their performance. We find
that the payload program's speculative
window is mainly limited by the CPU's reorder buffer, which
allows us to execute up to 200 instructions speculatively on modern
Intel CPUs. While brief, we show how to perform execution in short steps,
communicating intermediate results back to the ``real world'' part of the
payload program. Using this technique, we demonstrate implementing a universal
Turing machine (demonstrating arbitrary computation), a custom instruction set
architecture that fits within the constraints of speculative execution, and show
the ability to perform AES decryption using AES-NI instructions.

Using these building blocks, we demonstrate the practicality of hiding arbitrary
computation by implementing a reverse shell in our speculative instruction set,
with instructions decrypted in the speculative world.
We show that this simple payload is able to perform several system calls in a
reasonable time, ultimately launching a dial-back TCP shell in just over
2~milliseconds after the trigger is present.

%Using these building blocks, we demonstrate the practicality of hiding arbitrary
%computation by decrypting an AES-encrypted ARM binary in the speculative world
%one instruction at a time, returning the decrypted next instruction to the real
%world part of the payload program, which updates a virtual machine. We show that
%we are able to decrypt and process approximately 25 instructions per second.


% contributions?
% - explore the limits of speculative execution
% - propose novel concept of hiding computation in speculative execution
% - implement example applications using this concept, including decrypting
%   data speculatively
% - demonstrate using benign program (openssl) as trigger
% - identify defenses and discuss ways to counter them



% Move to discussion:
% TODO: we don't talk at all about how an analyst might try to enumerate
% the potential entry points to the speculative world, and how that can be
% made difficult




%%%%%%%%%%%%%%
