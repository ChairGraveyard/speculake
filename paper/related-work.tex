
\section{Related Work}
\label{sec:related-work}

% Many resourcess have been useful in developing the \speculake exploit 
% model.  

\subsection{Weird Machines \& Red-Pills}

%Hardware side effects often leave behind unexpected state transitions 
%that are not, or cannot be, accounted for in modeling environments. Malware
%has historically made use of these bugs to create "weird" models of 
%computation~\cite{weird_machines}. These state transitions can also leak 
%information about the environment in which a program is running, which 
%malware has used to detect honey-pots and emulated 
%environments~\cite{balzarotti2010efficient}.

\textit{Weird machines} are machines composing bugs and features of a system to
create arbitrary computation with interesting
properties~\cite{weird_machines,bratus2011exploit,weird_exploits}. This
encompasses many traditional computer vulnerabilities~\cite{buffer_overflow,
format_string_exploit, shacham2007geometry}. These exploits demonstrate
performing arbitrary computation in places that were not originally intended to
act as computational entities.

Unexpected functionality such as~\cite{bangert2013page} is abundant in modern
processors~\cite{d2015exploiting} as multiple threads share system resources
cooperatively and optimizations are often done across the process
isolation boundary. \speculake takes advantage of cooperative hardware
resources, and constructs a hardware weird machine
using speculative branching and \texttt{Flush+Reload}.

Hardware minutia have also been used to identify and fingerprint execution
environments---in malware this is know as a red-pill~\cite{red-pill} and is
typically used by malware to detect when it is under emulation or a debugging
environment~\cite{lindorfer2011detecting, balzarotti2010efficient,
paleari2009fistful}

% Hardware bare-metal analysis
To combat this Kirat~\textit{et al.} propose automated ``bare-metal''
emulation of malware~\cite{kirat2011barebox}, and methods to determine if malware
is performing red-pill checks to avoid emulation environments~\cite{kirat2014barecloud}.
Symbolic execution can also be used to combat red-pills by finding environmental triggers 
in malicious binaries~\cite{schwartz2010all}. 

Unlike other red-pills based on hardware bugs, \speculake cannot be bypassed
by patching binaries as the speculation will only occur on bare-metal systems, 
and the critical functionality is performed speculatively. Similarly, debugging 
break-points will not be triggered as instructions are never completed in a committed state.
Symbolic execution will also fail to find the speculative gadgets as it is never on the
execution pathway.

\subsection{Cache Side Channels}

Cache side-channels have been used against a number of processors at essentially
all different caching levels---L1 data
cache~\cite{percival2005cache,zhang2012cross,osvik2006cache}, L1 instruction
cache~\cite{aciiccmez2010new}, L2
cache~\cite{ristenpart2009hey,percival2005cache},
LLC~\cite{ristenpart2009hey,liu2015last}, and branch prediction
cache~\cite{aciiccmez2007power}. These attacks have also been performed in the
cloud~\cite{ristenpart2009hey,zhang2012cross},
browser~\cite{oren2015spy,google_cache_browser}, and security critical
encryption~\cite{yarom2014recovering,tromer2010efficient} environments,
demonstrating how ubiquitous hardware side-channel exploits can be. 

Cache side channels come in many
variants~\cite{neve2006refined,tromer2010efficient,yarom2014flush+,gruss2016flush+flush}.
In \speculake, we utilize the \texttt{Flush+Reload} method, where we first flush
all items from cache, wait for the speculative world to reload a value, and then
time accesses to reloading each potential value. However, we acknowledge that
the recent \texttt{Flush+Flush} variant may offer a faster
alternative~\cite{gruss2016flush+flush}.

%\begin{itemize}
%\item \texttt{Evict+Time} - An attacker with a synchronized clock evicts a series of items 
%from cache. The victim then proceeds with computation, at which point the attacker can 
%time accesses to each evicted item to identify which items a victim loaded~\cite{neve2006refined}. 
%\item \texttt{Prime+Probe} - An attaker performs a computation, then evicts specific 
%items from the cache before performing the same action again. Information about the cache
%access patterns can be gained by comparing the timimg of the two 
%computations~\cite{tromer2010efficient}.
%\item \texttt{Flush+Reload} - Two cooperating processes  make use of the \texttt{Evict+Time} 
%model to form a communication channel~\cite{yarom2014flush+}.
%\item \texttt{Flush+Flush} - Follows the same model as \texttt{Prime+Probe},
%however clflush is used instead of a load as it resolves much quicker in cases 
%where the item is not in cache~\cite{gruss2016flush+flush}.
%\end{itemize}


%While other works have demonstrated higher throughput from similar 
%covert cache side channels~\cite{liu2015last}, \speculake incurrs slightly higher performance 
%penalties in our implementation as we perform redundancy checks to 
%ensure the signal strength.  This redundancy combined with the intentional cache misseses
%which enable speculative execution cost hundreds to thousands of clock cycles
%per round. 


\subsection{Speculative Execution}

% Lead in 
Leading up to the discovery and release of the Spectre and Meltdown
vulnerabilities multiple teams were working to reverse engineer the mechanisms
that implemented in the microcode of various processors and document their
behavior~\cite{intel-instruction-tables,project_zero,measuring-cache,measuring-rob}.
Research into branch predictors~\cite{godbolt2016branch}, cache replacement
policies, and reorder buffer construction~\cite{measuring-rob} paved the way for
the branch prediction side-channels employed by Spectre (and subsequently
\speculake).
%Efforts to examine cache and memory security also preempted the Spectre work, 
%including Sophia D'Antoine's thesis work developing a side channel based on 
%instruction interleaving in the CPU~\cite{d2015exploiting}. Similar research into branch 
%predictors, cache replacement policies, and reorder buffer construction including work by Matt 
%Godbolt~\cite{godbolt2016branch}, and Henry Wong~\cite{measuring-rob} paved the runway for 
%dedicated branch prodiction side-channels.

% Discovery & Followup
Given the numerous directions that were being investigated in reverse
engineering Intel processors, multiple teams discovered the Spectre
vulnerabilities concurrently~\cite{spectre, project_zero, evtyushkin2018branchscope, maisuradze2018speculose}.
%These works propose multiple variants making use of branch prediction.
Follow 
up work on speculative execution side channels has also demonstrated
web-based vulnerabilites~\cite{genkin2018drive} as well as hardware isolation  
leaks in Intel's SGX~\cite{spectre_sgx}. At the time of writing multiple sources
have claimed new speculative execution related exploits termed \textit{Spectre-NG}
have been reported to Intel and assigned CVE numbers, though details are
not yet available~\cite{spectre_ng_gossip}.
%We have no details
%on content, though Intel has labeled four as critical and another four 
%as intermediate security flaws~\cite{spectre_ng_gossip}.  


% defenses

%Much like cache side channel vulnerabilities speculative execution is relatively
%critical to processor performance and requires significant software and hardware 
%updates to fully address. 
% Multiple measures have been proposed to combat the 
% individual variants of Spectre.

% One popular approach uses compiler level software patches to wrap sensitive 
% function calls. Retpoline replaces indirect jumps with \texttt{retq}
% instructions that read from the cached stack (and are thus less likely to be
% mis-speculated)~\cite{retpoline}.
% Gruss~\textit{et al.} propose the KAISER software patch to enforce address space
% isolation for kernel and user space proceses~\cite{gruss2017kaslr}. 
% Other proposals have considered hypervisor enforced time domain separation at varying 
% granularity~\cite{renau2018securing}, and fuzzy clocks that would reduce the accuracy 
% of any time based exfiltration pathway~\cite{hu1992reducing}. However, none of these defenses
% fundamentally defeat the use of a branch predictor as a cooperative mechanism
% spanning multiple processes. Intel has released microarchitecture updates for
% its most recent CPUs, but older architectures do not support these updates and
% will remain vulnerable~\cite{intel-micro-patch}.


% Trippel~\textit{et al.} proposes a method of faithfully simulating hardware
% bugs, given a formal model specifying the
% microarchitecture~\cite{trippel2018meltdownprime}. However, such models are
% unlikely to be released by CPU manufacturers, as they generally involve
% closely-guarded trade secrets, making this a less accessible solution for the
% research community.

%\smallskip
%
%We can certainly look forward to more vulnerabilities relating to speculative 
%branching and related hardware side channels. Our work represents an exploit 
%model for these vulnerabilities reliant upon shared hardware resources. 



%%%%%%%%%%%%%%
