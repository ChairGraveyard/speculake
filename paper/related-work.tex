\section{Related Work}
\label{sec:related-work}

% Many resourcess have been useful in developing the \speculake exploit 
% model.  

\subsection{Weird Machines}

\speculake shares many properties with \textit{weird machines}---a machine which
takes advantage of bugs or unexpected idiosyncracies in existing systems to
perform arbitrary
computation~\cite{weird_machines,bratus2011exploit}. In
particular \speculake showcases the ability to use CPU speculation to compute.

Recently there has been a trend of features in modern processors such as
multiple threads sharing system resources and optimizations done across the
process isolation boundary which lead to opportunities for ``weird
machines''~\cite{d2015exploiting}. In particular, previous examples of ``weird
machines'' include traditional vulnerabilities such as buffer overflows, format
string exploits and return oriented programming~\cite{buffer_overflow,
format_string_exploit, shacham2007geometry}. Weird machines have also been built
using operating system page faults, enabling the computation of arithmetic and
logic operations without the use of traditional
instructions~\cite{bangert2013page}.
%Additionally, unexpected functionality such as instruction-less computing
%enables the demonstration of a Turing complete ``weird machine'' by taking
%advantage of memory translation tables, page-faults, and hardware exception
%handling to effectively compute arithmetic and logical operations without the
%processor executing any instructions~\cite{bangert2013page}.
\speculake extends research in ``weird machines'', and takes advantage of
speculative execution to execute instructions that otherwise appear to be dead
code.


% Not sure this is true. It could be argued that Spectre is the first to do
% this, though we certainly leverage that for our purposes. It's a fuzzy
% distinction, but maybe best to leave this out.
%While previous weird machines have demonstrated the
%capability to execute potentially malicious code in an unexpected manner, we are
%the first to exploit speculative execution in order to execute arbitrary
%instructions.



\subsection{Covert Channels}

%programs~\cite{evtyushkin2016jump, lee2017inferring, aciiccmez2007predicting}
Spectre builds upon prior work on cache side channels, and similarly uses them
to leak information from
processes~\cite{percival2005cache,zhang2012cross,osvik2006cache}. In \speculake,
we use the branch predictor as a \emph{covert channel}~\cite{lampson1973note}
between the trigger program and malware payload, allowing the malware's
(speculative) execution path to be influenced by the trigger.

Previous work has examined how to share information over covert channels, such
as across virtualized environments on cloud systems~\cite{wu2012whispers}, using
L1 and L2 cache to share information~\cite{percival2005cache}, measuring
temperature to create a thermal covert channel~\cite{masti2015thermal,
bartolini2016capacity}, and taking advantage of processor architecture to leak
information~\cite{wang2006covert}. This includes using the branch predictor
itself as a covert
channel~\cite{evtyushkin2016understanding,evtyushkin2015covert}, which
\speculake similarly uses.

However, we note that the covert channel used in \speculake need not involve two
cooperative programs, and we demonstrate using the benign OpenSSL as a
non-colluding program involved in utilizing this covert channel.

% cache~\cite{aciiccmez2010new}, L2
% cache~\cite{ristenpart2009hey,percival2005cache},
% LLC~\cite{ristenpart2009hey,liu2015last}, and branch prediction
% cache~\cite{aciiccmez2007power}. These attacks have also been performed in the
% cloud~\cite{ristenpart2009hey,zhang2012cross},
% browser~\cite{oren2015spy,google_cache_browser}, and security critical
% encryption~\cite{yarom2014recovering,tromer2010efficient} environments,
% demonstrating how ubiquitous hardware side-channel exploits can be. 

% Cache side channels come in many
% variants~\cite{neve2006refined,tromer2010efficient,yarom2014flush+,gruss2016flush+flush}.
% In \speculake, we utilize the \texttt{Flush+Reload} method, where we first flush
% all items from cache, wait for the speculative world to reload a value, and then
% time accesses to reloading each potential value. However, we acknowledge that
% the recent \texttt{Flush+Flush} variant may offer a faster
% alternative~\cite{gruss2016flush+flush}.


\subsection{Speculative Execution}


\speculake builds on Spectre~\cite{spectre} and Meltdown~\cite{meltdown} which
leverage speculative execution to leak sensitive information from vulnerable
processes.  Follow up work has identified several new Spectre variants,
including speculative buffer overflows and speculative store
bypass~\cite{kiriansky2018speculative,spec-store-bypass}, and has investigated
additional ways to leak information using branch
predictors as a side channel~\cite{evtyushkin2018branchscope}.
Researchers have also leveraged Spectre and speculative execution more generally
to demonstrate web-based
vulnerabilities~\cite{genkin2018drive,schwarz2018netspectre} as well as to leak
control flow, keys, and other information from the hardware isolation provided by Intel
SGX~\cite{spectre_sgx,chen2018sgxpectre,foreshadow,lee2017inferring}.
Spectre has additionally been proposed as a way to thwart taint tracking by using speculative
execution to copy data between buffers~\cite{necessaryBlog}.
\speculake likewise takes advantage of speculative execution, but with the goal
of hiding arbitrary computation from reverse engineering, rather than extracting
secrets from vulnerable programs. \speculake also benefits from
new Spectre variants: as we showed, speculative buffer overflows (``Spectre~1.1'')
can be used as an alternative trigger for malware.


