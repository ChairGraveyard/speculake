
\section{Related Work}
\label{sec:related-work}

% Many resourcess have been useful in developing the \speculake exploit 
% model.  

\subsection{Weird Machines \& Red-Pills}
Hardware side effects often leave behind unexpected state transitions 
that are not, or cannot be, accounted for in modeling environments. Malware
has historically made use of these bugs to create "weird" models of 
computation~\cite{weird_machines}. These state transitions can also leak 
information about the environment in which a program is running, which 
malware has used to detect honey-pots and emulated 
environments~\cite{balzarotti2010efficient}.

The term \textit{weird machine} comes from work done by Bratus \textit{et al.}
composing bugs and features of a system to create arbitrary compuatation with 
interesting properties~\cite{weird_machines,bratus2011exploit}. This 
encompasses many traditional computer vulnerabilities including 
buffer overflow~\cite{buffer_overflow}, format string exploits~\cite{format_string_exploit}, 
and return oriented programming~\cite{shacham2007geometry} to name only a few. 

Bangert~\textit{et al.} demonstrate that a turing complete weird machine 
can be constructed using the page fault handling mechanism in Intel's
IA32 architecture~\cite{bangert2013page}. Bugs like this are abundant in
modern processors~\cite{d2015exploiting} as multiple threads share system resources cooperatively
and optimizations is done objectively with regards of the 
process isolation for the most part~\footnote{With the exception of permission 
rings, i.e. kernel vs user space isolation}. Accessing cooperative hardware resources is at
the heart of the \speculake model, which constructs a hardware weird machine
using speculative branching and \texttt{Flush+Reload}. This is 
very much in the spirit of traditional exploit creation~\cite{weird_exploits}.

\smallskip 

Similarly hardware bugs have been used to identify and fingerprint execution
environments -- in malware this is know as a red-pill~\cite{red-pill} and 
is typically used to ensure that malware is not being emulated or 
debugged~\cite{lindorfer2011detecting, balzarotti2010efficient}. Paleari 
\textit{et al.} automate the process of discovering red-pills by applying 
fuzzing techniques to emulated environments and identifying situations 
in which the emulators differ from their physical counterparts~\cite{paleari2009fistful}.
This is still by no means an exhaustive list of red-pills for a given 
processor as many rely on timing or race conditions in physical 
environments. 

% Hardware bare-metal analysis
To combat this Kirat \textit{et al.} propose automated "bare-metal" 
emulation of malware~\cite{kirat2011barebox}, and methods to determine if malware
is performing red-pill checks to avoid emulation environments~\cite{kirat2014barecloud}.
Symbolic execution can also be used to combad red-pills by finding environmental triggers 
in malicious binaries~\cite{schwartz2010all}. 

\smallskip

Unlike other red-pills based on hardware bugs, \speculake cannot be bypassed
by patching binaries as the speculation can only happen on "bare-metal" systems, 
and the critical functionality is performed speculatively. Similarly, debugging 
break-points wll not be triggered as instructions are never completed in a committed state.
Symbolic execution will also fail to find a target function as it is never on the 
execution pathway. 



\subsection{Cache Side Channels}

The \speculake model relies critically on cache side channel attacks to 
communicate information from the speculative world. Cache side-channels have 
been used against a number of processors at essentially all different 
caching levels -- L1 data cache~\cite{percival2005cache,zhang2012cross,osvik2006cache}, 
L1 instruction cache~\cite{aciiccmez2010new}, 
L2 cache~\cite{ristenpart2009hey,percival2005cache}, 
LLC~\cite{ristenpart2009hey,liu2015last}, 
and branch prediction cache~\cite{aciiccmez2007power}. 
These attacks have also been performed in 
cloud~\cite{ristenpart2009hey,zhang2012cross}, 
browser~\cite{oren2015spy,google_cache_browser},
and security critical encryption~\cite{yarom2014recovering,tromer2010efficient}
environments, demonstrating how ubiquitous hardware side-channel exploits can be. 

Most cache side-channles fall into one of the following categories, although
some implement variations on these models.

\begin{itemize}
\item \texttt{Evict+Time} - An attacker with a synchronized clock evicts a series of items 
from cache. The victim then proceeds with computation, at which point the attacker can 
time accesses to each evicted item to identify which items a victim loaded~\cite{neve2006refined}. 
\item \texttt{Prime+Probe} - An attaker performs a computation, then evicts specific 
items from the cache before performing the same action again. Information about the cache
access patterns can be gained by comparing the timimg of the two 
computations~\cite{tromer2010efficient}.
\item \texttt{Flush+Reload} - Two cooperating processes  make use of the \texttt{Evict+Time} 
model to form a communication channel~\cite{yarom2014flush+}.
\item \texttt{Flush+Flush} - Follows the same model as \texttt{Prime+Probe},
however clflush is used instead of a load as it resolves much quicker in cases 
where the item is not in cache~\cite{gruss2016flush+flush}.
\end{itemize}

While other works have demonstrated higher throughput from similar 
covert cache side channels~\cite{liu2015last}, \speculake incurrs slightly higher performance 
penalties in our implementation as we perform redundancy checks to 
ensure the signal strength.  This redundancy combined with the intentional cache misseses
which enable speculative execution cost hundreds to thousands of clock cycles
per round. 


\subsection{Speculative Execution}

% Lead in 
Leading up to the discovery and release of the Spectre and Meltdown vulnerabilities
multiple teams were working to reverse engineer the mechanisms that implemented in 
the microcode of various processors and document their behavior~\cite{intel-instruction-tables, uop_article}.
Efforts to examine cache and memory security also preempted the spectre work, 
including Sophia D'Antoine's thesis work developing a side channel based on 
instruction interleavings in the CPU~\cite{d2015exploiting}. Similar research into branch 
predictors, cache replacement policies, and reorder buffer construction including work by Matt 
Godbolt~\cite{godbolt2016branch}, and Henry Wong~\cite{measuring-rob} paved the runway for 
dedicated branch prodiction side-channels.

% Discovery & Followup
Given the numerous directions that were being investigated in reverse engineering 
Intel processors, multiple teams discovered the Spectre vulnerabilites concurrently including 
the teams associated with Kocher \textit{et al.}~\cite{spectre}, the Project Zero team 
at Google~\cite{project_zero},and various others~\cite{evtyushkin2018branchscope,maisuradze2018speculose}. 
These works propose multiple variants making use of banch prediction.  Follow 
up work on speculative execution side channels have also demonstrated
web-based vulnerabilites~\cite{genkin2018drive} as well as hardware isolation  
leaks in Intel's SGX~\cite{spectre_sgx}. 


% defenses 
Much like cache side channel vulnerabilities speculative execution is relatively
critical to processor performance and requires significant software, and hardware 
updates to fully address. Multiple measures have been proposed to combat the 
individial variants of Spectre.   

The main defenses that have been widely applied to 
speculative branching attacks uses compiler level software patches to wrap sensitive 
function calls. Retpoline is Google's in house fix for enforcing 
isolation against branch target injection attacks~\cite{retpoline}. By crafting 
the stack around a sensitive function to trap into a pause instruction if the processor 
attempts to speculate past the wrapper data side-effects can be protected against. 
Similarly, Gruss \textit{et al.} propose the KAISER software patch to enforce address space
isolation for kernel and user space proceses~\cite{gruss2017kaslr}. 
Other proposals have considered hypervisor enforced time domain separation at varying 
granularity~\cite{renau2018securing}, and fuzzy clocks that would reduce the accuracy 
of any time based exfiltration pathway~\cite{hu1992reducing}. However, none of these defenses 
fundamentally defeat the use of a branch predictor as a cooperative mechanism
spanning multiple processes. 

Trippel \textit{et al.} proposes a method of faithfully simulating 
hardware bugs, such as processes cooperatively using speculating branch prediction,
by synthesizing microarchetecture defects~\cite{trippel2018meltdownprime}. However, 
manufacturers generally guard complete formal descriptions of processor microarchitecture closely, 
making this a less accessible solution for the research community.

\smallskip

We can certainly look forward to more vulnerabilities relating to speculative 
branching and related hardware side channels. Our work represents an exploit 
model for these vulnerabilities reliant upon shared hardware resources. 



%%%%%%%%%%%%%%
