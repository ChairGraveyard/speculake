
\begin{abstract}
Recently, the Spectre and Meltdown attacks revealed serious vulnerabilities
in modern CPU designs, allowing an attacker to exfiltrate data from sensitive
programs. These vulnerabilities take advantage of speculative execution to
coerce a processor to perform computation that would otherwise not occur,
leaking the resulting information via side channels to an attacker.

In this paper, we extend these ideas in a different direction, and leverage
speculative execution in order to \emph{hide malware} from both static and
dynamic analysis. Using this technique, critical portions of a malicious
program's computation can be shielded from view, such that even a debugger
following an instruction-level trace of the program cannot tell how its results
were computed.

We introduce \emph{\speculake}\footnote{Executable Spectre}, which compiles
arbitrary malicious code into a seemingly-benign payload binary. When a separate
trigger program runs on the same machine, it mistrains the CPU's branch
predictor, causing the payload program to \emph{speculatively} execute its malicious
payload, which communicates speculative results back to the rest of the payload program
to change its real-world behavior.

We study the extent and types of execution that can be performed
speculatively, and demonstrate several computations that can be
performed covertly. In particular, within speculative execution we are able to
decrypt memory using AES-NI instructions at over 5~kbps. Building on this, we
decrypt and interpret
a custom virtual machine language to perform arbitrary computation and system
calls in the real world. We demonstrate this with a proof-of-concept dial back shell,
which takes only a few milliseconds to execute after the trigger is issued.
We also show how our corresponding trigger program
can be a pre-existing benign application already running on the system, and
demonstrate this concept with OpenSSL driven remotely by the attacker as a
trigger program.

\speculake demonstrates a new kind of malware that evades existing reverse
engineering and binary analysis techniques. Because its true functionality is
contained in seemingly unreachable dead code, and its control flow driven
externally by potentially any other program running at the same time, \speculake
poses a novel threat to state-of-the-art malware analysis techniques.




%We consider this to be a new thrust in attack vector research that harkens to work 
%done on hardware red-pills and weird machines. We demonstrate that unintended 
%functionality of architechture can be used to compose malicious programs with 
%new and different properties. This technique poses novel and unique challenges for 
%malware modeling efforts, as it forces an environment to either
%faithfully reproduce all hardware behaviors, or overcome a much larger burden of 
%reverse engineering. 


\end{abstract}

